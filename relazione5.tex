\documentclass{article}

\title{Relazione 4: Misura del passo di un reticolo di diffrazione e di lunghezze d'onda}
\date{}

\begin{document}
		
	\maketitle
	
	Scopo della presente relazione è verificare la legge di Malus e misurare la concentrazione di una soluzione che presenta attività ottica. L'apparato sperimentale è costituito da un banco ottico a sezione triangolare sul quale possono essere montati e fatti scorrere dei cavalieri. Su due di questi sono montati dei filtri polarizzatori su supporti ruotanti che presentano una scala goniometrica con sensibilità $S=1^{\circ}/div$. Alle estremità del banco ottico si trovano: da una parte un laser a diodo che emette luce di lunghezza d'onda $\lambda=650nm$; dall'altra una fotocellula, che eroga una corrente continua di intensità $I$ proporzionale alla potenza luminosa che incide sulla sua area sensibile. La fotocellula è collegata a un amperometro digitale che, per correnti di intensità inferiore a $1mA$, fornisce misure con incertezza allargata pari a $\Delta I = (0,01 I \pm 1 digit)$.
	Al fine di verificare la legge di Malus, si accende il laser e si posiziona sul banco ottico il primo polarizzatore; quindi, si ruota il supporto di quest'ultimo fino a ottenere la massima intensità di corrente misurata dall'amperometro. La misura della posizione angolare assoluta del polarizzatore che corrisponde a questa condizione è $\Theta_P=(\pm)^{\circ}$. Si posiziona sul banco anche il secondo polarizzatore e se ne ruota il supporto fino a massimizzare il valore di I misurato dall'amperometro. La misure della posizione angolare assoluta dell'analizzatore quando ciò si verifica è $\Theta_{a,0}=(\pm)^{\circ}$. Quindi, si misurano i valori di $I$ corrispondenti alle distanze angolari $\alpha$ da $\Theta_{a,0}$ multiple di $10^{\circ}$ fino ad $\alpha=90^{\circ}$ per la quale si ottiene il valore minimo di I. Tali misure sono riportate in Tabella 1, dove sono anche indicati i valori di $\cos^2\alpha$. Sul Grafico 1 è riportato l'andamento di $I$ in funzione di $\cos^2\alpha$, il quale appare lineare. I coefficienti della retta di best-fit valgono $A=(\pm)$ e $B=(\pm)$.
	
	[magari dire da qualche parte che i supporti sono stati messi il più vicino possibile alla fotocellula per essere sicuri che il laser incidesse bene sulla zona sensibile etc..]
	
\end{document}