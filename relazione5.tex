\documentclass{article}

\title{Relazione5: Verifica della legge di Malus e misura della concentrazione di una soluzione zuccherina}
\date{}

\begin{document}
		
	\maketitle
	
	Scopo della presente relazione è verificare la legge di Malus e misurare la concentrazione di una soluzione che presenta attività ottica. 
	
	L'apparato sperimentale è costituito da un banco ottico a sezione triangolare sul quale possono essere montati e fatti scorrere dei cavalieri. Su due di questi sono montati dei filtri polarizzatori su supporti ruotanti che presentano una scala goniometrica con sensibilità $S=1^{\circ}/div$. Alle estremità del banco ottico si trovano: da una parte un laser a diodo che emette luce di lunghezza d'onda $\lambda=650nm$; dall'altra una fotocellula, che eroga una corrente continua di intensità $I$ proporzionale alla potenza luminosa che incide sulla sua area sensibile. La fotocellula è collegata a un amperometro digitale che, per correnti di intensità inferiore a $1mA$, fornisce misure con incertezza allargata pari a $\Delta I = (0,01 I \pm 1 digit)$.
	
	Al fine di verificare la legge di Malus, si accende il laser e si posiziona sul banco ottico il primo polarizzatore; quindi, si ruota il supporto di quest'ultimo fino a ottenere la massima intensità di corrente misurata dall'amperometro. La misura della posizione angolare assoluta del polarizzatore che corrisponde a questa condizione è $\Theta_P=(\pm)^{\circ}$.
	La luce così ottenuta risulta [avere un alto grado di polarizzazione lineare] linearmente polarizzata.
	
	Si posiziona sul banco anche il secondo polarizzatore e, dopo essersi assicurati che la luce proveniente dal laser incidesse sulla zona sensibile della fotocellula, se ne ruota il supporto fino a massimizzare il valore di I misurato dall'amperometro. La misure della posizione angolare assoluta dell'analizzatore quando ciò si verifica è $\Theta_{a,0}=(\pm)^{\circ}$. Quindi, si misurano i valori di $I$ corrispondenti alle distanze angolari $\alpha$ da $\Theta_{a,0}$ multiple di $10^{\circ}$ fino ad $\alpha=90^{\circ}$ per la quale si ottiene il valore minimo di I. Tali misure sono riportate in Tabella 1, dove sono anche indicati i valori di $\cos^2\alpha$. Sul Grafico 1 è riportato l'andamento di $I$ in funzione di $\cos^2\alpha$, il quale appare lineare. I coefficienti della retta di best-fit valgono $A=(\pm)$ e $B=(\pm)$.
	
	[magari dire da qualche parte che i supporti sono stati messi il più vicino possibile alla fotocellula per essere sicuri che il laser incidesse bene sulla zona sensibile etc..]
	
	
	Assicuratisi del buon funzionamento dei polarizzatori e della validità della legge di Malus nel descrivere il loro comportamento, si vuole passare allo studio dell'attività ottica di una soluzione zuccherina.
	In particolare, si vuole misurare la rotazione $\Delta \theta$ del piano di oscillazione del campo elettrico della luce passante attraverso un tubo polarimetrico riempito con la soluzione e utilizzarla per ricavare la concentrazione del D-Saccarosio in tale soluzione, noto il valore del suo potere rotazionale $\rho(\lambda=650nm)=...$.
	
	Per misurare la rotazione $\Delta \theta$, si dispone prima l'apparato sperimentale con i polarizzatori in posizione incrociata, ruotando il secondo polarizzatore in modo da minimizzare la corrente erogata dalla fotocellula. In tale posizione, gli assi dei polarizzatori risultano ortogonali tra loro.
	
	Si introduce quindi il tubo polarimetrico, montato su un apposito cavaliere, in una posizione intermedia tra i due polarizzatori [assicurandosi che la rifrazione della luce al suo interno non devi il fascio fuori dall'area sensibile].
	Come atteso, data l'attività ottica della soluzione zuccherina, si osserva che il valore di I misurato dall'amperometro aumenta sensibilmente dopo l'introduzione del tubo polarimetrico.
	Si ruota quindi il secondo polarizzatore fino a trovare un nuovo minimo dell'intensità luminosa rilevata dalla fotocellula, rendendo l'asse del polarizzatore ortogonale alla nuova direzione del campo elettrico della luce e quindi ruotandolo nel processo di un angolo pari a $\Delta \theta$. 
	
	Effettuando con questa tecnica misure ripetute dell'angolo $\Delta \theta$ per tre tubi polarimetrici diversi (Riportate in Tabella x), è possibile risalire alla concentrazione della soluzione tramite la formula:
	$$c = \frac{\Delta \theta}{\rho L}$$
	
	dove compare la lunghezza del tubo polarimetrico [tecnicamente la lunghezza del percorso della luce attraverso la soluzione, che assumiamo essere pari a quella del tubo], che è stata misurata mediante un regolo di sensibilità $1 div/mm$.
	Utilizzando i valori $L_1 = (1.07 \pm 0.01) dm$, $L_2 = (1.09 \pm 0.01) dm$ e $L_3 = (1.09 \pm 0.01) dm$ per i tre tubi analizzati, si ricavano dunque le concentrazioni ....
	[risultano consistenti tra loro? sembra di si, quindi dovremmo fare la media pesata per dare un unico valore]
	

	
\end{document}